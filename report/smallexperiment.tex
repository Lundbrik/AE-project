\section{Empirical investigation}
The sorting networks have been collected in a method that selects a prober
sized network for the input. These methods has been tested on an environment
with the following specification\\

\begin{figure}
    \centering
    \begin{tabular}{ll}
            \textit{OS:} & Linux 64-bit\\
            \textit{Version:} & 4.8.4-1-ARCH\\
            \textit{Processor:} & Intel i5 2.67 GHz\\
            \textit{RAM:} & 12 GB\\
            \textit{Compiler:} & GNU, G++, 6.2.1
    \end{tabular}
    \caption{Benchmark machine 1}
    \label{fig:pc1}
\end{figure}

The methods has been averaged over a set of runs on a variation of smaller set
sizes. The result is depicted in figur~\ref{smalltablefigure}. It is clear from
the results that the branch optimization of Ford \& Johnson does not make a
significant impact. This can be due to a lot of reasons, but most likely stems
from the fact that the few branches in the implementation does not contribute
that much to the runtime. It is evident that the Ford \& Johnson algorithm may
be close to comparison optimal, but in practice the swapping and shifting of
elements of which it is fairly dependent is too slow. Further more the
specialized sorting networks are significantly faster than Ford \& Johnson even
though they achieve almost the same number of comparisons and are aslo
dependent on swapping elements.

Compared to the standard library sorting method, it is clear that the
specialized sorting networks achieve almost the same runtime. The standard
library of the gnu gcc compiler uses a hybrid between introsort, itself being a
hybrid, and insertion sort\cite{gnustd}. As described the sorting networks
achieves around the same runtime as this hybrid method on smaller set sizes.

\begin{figure}
\begin{tabular}{rrrrrrrrr}
    $N$ & \ttt{std::sort}   & \ttt{nsort()} & \ttt{fsort()} & Ford Johnson  & FJ branch optimized\\\hline
    2   & 12                & 9             & N/A           & 110       & 122\\
    4   & 32                & 21            & 21            & 247       & 319\\
    6   & 57                & 38            & 37            & 501       & 673\\
    8   & 89                & 60            & 60            & 603       & 683\\
    10  & 119               & 85            & N/A           & 895       & 1291\\
    12  & 154               & 115           & N/A           & 1018      & 1511\\
    14  & 192               & 147           & N/A           & 1375      & 2122\\
    16  & 233               & 148           & N/A           & 1609      & 2398\\
\end{tabular}
    \caption{Comparison of small sets specialized sortes and \ttt{std::sort}.
    \ttt{nsort()} is the sorting networks without branch optimization (due to
    \ttt{g++} not outputing \ttt{cmov}s). \ttt{fsort()} is sorters with optimal
    number of comparisons for $N$ elements. Machine see Figure \ref{fig:pc1}}
\label{smalltablefigure}
\end{figure}
