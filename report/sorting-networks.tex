\section{Sorting networks}
Sorting networks, are networks with some number of inputs, that then using
connections for comparison and swaps, sort the inputs. They are usually
depicted as seen in Figure \ref{fig:sort-network1}, where the horizontal lines
are the inputs, and the vertical lines are the comparators, the comparators
compare the two inputs that it is connected to, and then swap the values if the
value for the lower input is lower than the value for the upper input.
\begin{figure}[htb]
    \centering
    \begin{sortingnetwork}{4}{0.4}
        \nodeconnection{ {1,2}, {3,4}}
        \addtocounter{sncolumncounter}{2}
        \nodeconnection{ {1,3}}
        \nodeconnection{ {2,4}}
        \addtocounter{sncolumncounter}{2}
        \nodeconnection{ {2,3}}
    \end{sortingnetwork}
    \caption{4-input}
    \label{fig:sort-network1}
\end{figure}

\subsection{Implementation}
We have implemented sorting networks for inputs from 2 up to 16 in
\ttt{nsort.h}, with the ones for inputs 2 to 9 being optimal
\cite{CCFS-2014}\cite{ron-zeno} in length (number of comparators). The rest,
inputs 10 to 16 was the best we could find with respect to length
\cite{hugues-juille}\cite{ron-zeno}. Also to have a nice drawing of them all
they are presented in \ref{app:sort-networks}.

\subsection{Bitonic mergesort}
Other than static sorting networks there also exists algorithms to generate
sorting networks, one such algorithm is Ken Batchers Bitonic mergesort. We
implemented the traditional version of the Bitonic sorter, which only works on
input sizes of $n = 2^k$ for some $k$, though it is possible to extend it to
accept arbitrary $n$ inputs \cite{hw-lang}. The implementation is present in
\ttt{bitonic.h}, is non-recursive, uses the alternative representation thus
only sorts in one direction, and has been tested with random input of different
$n = 2^k$ lengths inputs.

\subsection{Branch optimization}
Since practically all a sorting network does is comparing values and swapping
if needed, and ignoring the network is oblivious to previous branches, it lends
itself to use the \ttt{cmov} instruction that is available in todays x86 CPUs
\cite{CCNS-2015}. To test the viability of this optimization we created
\ttt{nsort.c} which is simply the 16 input sorting network and a main function
to run the sorter on a specified number of size 16 arrays of random values, and
finally print the total time in microseconds it took to sort them all, with the
possibility to change the comparator macro (in code), between the one that
should generate \ttt{cmov}s and the one that should generate jumps.

\begin{center}
    16-input sorting network over $10^5$ size 16 arrays.\\
    \begin{tabular}{lr}
        Type             & Time ($\mu$s)\\\hline
        Branch optimized & 5 854\\
        Regular          & 18 433\\
    \end{tabular}
\end{center}

The time difference makes it quite clear that the branch optimization is
non-negligible. And looking at brach misprediction with \ttt{valgrind} and
\ttt{cachegrind} we see why:

\begin{center}
    16-input sorting network over $10^5$ size 16 arrays.\\
    \begin{tabular}{lrrr}
        Type             & Branches & Mispredicts & Rate(\%) \\\hline
        Branch optimized & 19,334,579& 155,776    &  0.8\%\\
        Regular          & 25 334 579& 2,481,889  &  9.8\%\\
    \end{tabular}
\end{center}

However during testing a peculiarity of the \ttt{g++} optimizer came forth, if
we compile \ttt{nsort.c} with \ttt{g++} instead of \ttt{gcc} but still the same
flags then it will not optimize the comparators into \ttt{cmov} instructions.
Quick to see by running \ttt{make nsort-c; make nsort-cpp} and then run
\ttt{objdump -M intel -d build/nsortbo | grep cmov | wc -l} we get an output of
\ttt{120} whichs seem reasonable enough but
\ttt{objdump -M intel -d build/nsort | grep cmov | wc -l} gives an output of
\ttt{0}. All this would make it seem that \ttt{g++} simply does not have the
optimization that \ttt{gcc} has, but after implementing Bitonic mergesort we of
course wanted to see what improvements the same branch optimization would have
on the implementation, and quite unexpectedly we found that in this case
\ttt{g++} would employ the \ttt{cmov} instruction as is evident when we run
(Make sure the right macro is uncommented) \ttt{make bitonic} and afterwards
\ttt{objdump -M intel -d build/bitonic | grep cmov | wc -l} and get the output
\ttt{29}.




\subsection{Testing}
Since it can be quite hard to see from the code if a specific sorting networks
should work, testing them is quite important. So for all the implemented
sorting networks for inputs $>4$ there is a test for a reverse sorted list, and
for a number of runs on random inputs. Futhermore there is the possibility to
run a full permutation test, that is testing the sorting networks for all
possible permutations of inputs of that size, though this should be possible to
make in $O(2^n)$ instead of the current $O(n!)$ due to the Zero-one principle.
This permutation test has been run on all the implemented sorting networks up
to and including \ttt{nsort\_13()}
