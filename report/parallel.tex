\section{Parallel}
In this section, we will be looking at parallelising sorting algorithms. We
will be looking at sorting elements residing within the \texttt{vector} data
structure. Since we are trying to shave off time, we start by looking into
different approaches to memory management, allocation and access. From there we
will be implementing sequential versions of Quicksolt and Mergesort. Then we
will be parallelising these algorithms, and finally combine a combination of
both.

\subsection{Approaches to memory allocation, management and access}
Here we will be looking at inserting into a new empty array, moving data from
one \texttt{vector} to another, moving data from one \texttt{vector} into an
empty one and extracting a subset \texttt{vector} from a \texttt{vector}.

\subsubsection{Memory allocation}
Within the different algorithms, we sometimes want to insert one element at a
time into a \texttt{vector}. We have two approaches: calling \texttt{vector.insert()}
one element at a time, and calling \texttt{vector.push\_back()}. For each we look at
the performance with and without calling \texttt{vector.reserve()}.

We have \texttt{vector.insert()}:
\begin{center}
aoeu!
\end{center}
Here we see that calling \texttt{vector.reserve()} prior to the operations benefit the
performance.

Then, we have \texttt{vector.push\_back()}:
\begin{center}
aoeu!
\end{center}
Again, we see that calling \texttt{vector.reserve()} prior to the operations benefit the
performance.

To make sure, we have the graph containing both:
\begin{center}
aoeu!
\end{center}
From this graph, we see that calling \texttt{vector.push\_back()} gives the
best performance.

\subsubsection{Memory management and access}
When we want to move data from one \texttt{vector} to another, not empty,
\texttt{vector}. In these tests, we move a copy of a \texttt{vector} into the
original \texttt{vector}. We have following three approaches: manual move using
a for loop and using \texttt{vector.insert()}. For the manual move, we have
where we call \texttt{vector.size()} in the loop condition, and where we store
it in a variable before the loop. For insert, we have where we do and do not
call \texttt{vector.reserve()}.

Manual move:
\begin{center}
aoeu
\end{center}
Here, we see that moving the call to \texttt{vector.size()} clearly improves
the performance.

\texttt{vector.insert()}:
\begin{center}
aoeu
\end{center}
strangely enough, even though we are moving a copy, calling
\texttt{vector.reserve()} beforehand improves performance.

Finally, we compare the two best:
\begin{center}
aoeu
\end{center}
Here, we see that the manual move clearly beats the call to
\texttt{vector.insert()}

\subsection{Quicksort}
Quicksort is an algorithm, which is known to be good in practice. The
performance of the algorithm does depend on the quality of the split, i.e. if
the splits are balanced. We have made different versions of Quicksort, and
modified them based on observations when running them.

\subsubsection{Sequential}
To start off, we implemented an in-place Quicksort, which we would then
parallize. To pick the pivot element, we just pick one at random.

\subsubsection{Naive parallel implementation}
The naive parallization, is to for each split, spawn a thread sorting each
split. This however, has a far too big overhead, which results in error on
larger input, as too many threads are spawned. On the smaller input, the
overhead from spawning threads is far greater than the gains from parallizing,
resulting in very bad performance.

\subsubsection{Limited parallel implementation}
Following the naive implementation, we tried to make the same approach, but
only spaw a limited number of threads. This did give a speedup, on larger
input. However, the performance of this approach depends a lot on the splits.
E.g. if we get a bad split in the beginning, i.e. one small portion and one big
portion, the thread getting the smaller portion finishes faster, and does not
do any work afterwards.

\subsubsection{Modified limited parallel implementation}
Following the idle threads from the previous implementation, we thought of
another way. We have a global integer variable, which denotes the maximum
number of threads. Each time a thread is spawned, the integer is decremented.
Every time a thread is done, the integer is incremented. As such, if a thread
is done with its part earlier than others, one of the threads with more work,
can spawn more threads, thus keeping the processor fed. The overhead compared
to the limited parallel is not very big.

\subsubsection{Thread pool approach}
Another approach to keeping the processor at work at all times, are a thread
pool approach. In this approach we have a queue and an array of threads. Each
thread tries to get a "job", which is two indices, indicating the start and the
end of the split to sort.

One problem we faced was when to stop the threads, as we could not use an empty
queue, as the queue starts empty, and there are no real way to know that the
entire input is sorted. The approach we took was to have a boolean value,
indicating whether or not the bottom has been reached. Once it has, and once
the queue is empty, threads return.

This approach keeps the processor at a high load. However, the overhead
introduced from creating, and synchronizing, i.e. mutex locking and unlocking,
is very large, so this approach only works on larger input.

\subsection{Mergesort}
Mergesort is an algorithm, which is known to have very good performance when
having parallization. This is due to the equal splitting all the way through
the algorithm, which is always perfectly balanced. The hard part of this
algorithm was to make it in-place. We took a different approach: when we merge,
we copy one half of the input into a temporary vector, merge the temporary
vector and the other half into the original input.

\subsection{Sequential}
We started by making a normal sequential implementation. This implementation
worked very well, both against Quicksort and \texttt{std::sort()}.

\subsection{Limited parallel}
We took the same initial approach to parallizing as in the limited parallel
Quicksort, i.e. spawing only a limited number of threads. Due to the balance of
Mergesort, the processor should be at work all the time.

\subsection{Quickmerge}
We also thought of having Mergesort at the top of the input, and on each split
spawn a limited number of threads. This would remove the chance of bad splits
in quicksort, as all of the threads receive equal amount of work.

